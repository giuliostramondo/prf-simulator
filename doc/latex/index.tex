The Polymorphic Register File (P\+R\+F) are memory modules designed in order to allow fast parallel access to matrices in high performance applications. Through the use of mapping functions ( \hyperlink{prf_8h_a6177f3af58c89af5677ffc752aad36f8}{m\+\_\+v(int,int,scheme,int,int)}, \hyperlink{prf_8h_a6e7fdfb836e4314dda58fbc4be598bc0}{m\+\_\+h(int,int,scheme,int,int)}, \hyperlink{prf_8h_a7b0fcc330f84ed919cb9d30ae0b6dfeb}{A\+\_\+standard(int,int,int,int)} ) an N-\/\+Dimensional matrix is stored in an N+1-\/\+Dimensional structure enabling, for certain type of matrix accesses, a faster retrival of the data. The additional dimension in fact should contain N-\/\+Dimensional matrices, part of the original matrix, ordered following the sequence of accesses done by the application that is being accelerated. The data contained in the N-\/\+Dimensional sub-\/matrices can then be read in parallel in each computation step. There are multiple ways in which the sub-\/matrices can be organized, each one best suited for specific acces pattern\+:

\begin{TabularC}{2}
\hline
\rowcolor{lightgray}{\bf Access Scheme }&{\bf Description  }\\\cline{1-2}
Rectangle only &Generates rectangular patches from the original matrix. \\\cline{1-2}
Rect\&Row &Generates rectangular patches from the original matrix and swaps the rows. \\\cline{1-2}
Rect\&Col &Generates rectangular patches from the original matrix and swaps the columns. \\\cline{1-2}
Row\&Col &??? \\\cline{1-2}
Rect\&Trect &??? \\\cline{1-2}
\end{TabularC}
This simulator was written to allow a easier visualization of the effect of each access scheme and to provide a platform for the exploration of custom access schemes.

The simulator is implemented following the description given in \href{http://ieeexplore.ieee.org/xpls/abs_all.jsp?arnumber=6322873&tag=1}{\tt \char`\"{}\+On implementability of Polymorphic Register Files\char`\"{}}.

\section*{Installation }

The sources for the program are available on git ( once obtained the permission from the owner of the repository ).

To download the code execute the following line from the terminal.


\begin{DoxyCode}
1 git clone https://github.com/giuliostramondo/prf-simulator.git
\end{DoxyCode}


To compile the sources 
\begin{DoxyCode}
1 cd PRF\_Simulator
2 make
\end{DoxyCode}


This should produce an executable called prf.

\section*{Usage }

Usage\+: ./prf \mbox{[}Options\mbox{]}

-\/\+N $<$num$>$ Change the horizontal size of the input matrix (default 9)

-\/\+M $<$num$>$ Change the vertical size of the input matrix (default 9)

-\/p $<$num$>$ Change the horizontal size of the P\+R\+F (default 3)

-\/q $<$num$>$ Change the horizontal size of the P\+R\+F (default 3)

-\/s $<$num$>$ Change the schema used by the P\+R\+F (default 0 -\/$>$ R\+E\+C\+T\+A\+N\+G\+L\+E\+\_\+\+O\+N\+L\+Y) other schemes 1-\/$>$R\+E\+C\+T\+\_\+\+R\+O\+W, 2-\/$>$R\+E\+C\+T\+\_\+\+C\+O\+L, 3-\/$>$R\+O\+W\+\_\+\+C\+O\+L, 4-\/$>$R\+E\+C\+T\+\_\+\+T\+R\+E\+C\+T 